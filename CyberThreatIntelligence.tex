\documentclass{article}
\usepackage[utf8]{inputenc}
\usepackage{graphicx}
\graphicspath{ {C:\Users\Gopi\Desktop\sonyimage.jpg} }

\title{Cyber Threat Intelligence}
\author{Gopichandu.Para }
\date{11-February 2016}

\begin{document}

\maketitle

\section{Introduction}
This paper provides information about the Cyber Threat Intelligence. As in the today's world security is the major concern because every thing is based on the Internet. There are a lot of advantages on one side and at the same time we have threats on the other side. We have a lot of applications and lot of data stored in those applications which has to be managed very securley. Some how we have to protect this data from the attackers with a possible solution.

\section{What is cyber threat?}
The high level process of damaging the computer network by introducing attacks which leads to abnormal behaviour.Threat is nothing but an attack form the third person in the network who can gather all the information which has to be stored securely.Cyber criminals are becoming more powerful by establishing traditional methodologies which are faster than the organizations who can protect them.

\section{Information security}
Department of Homeland Security (DHS) has developed and executed a lot of programs regarding information sharing.DHS also shares information with government as the attackers are not confined to geographic boundaries. The attacks are still going on even the data is encrypted. So we should not leave the data unattended.

\section{Recent Cyber attack on Sony}
Sony faced a cyber attack recently last November which is a shock to them revealing a skeleton image on the computer screen as soon as it turned on with a message "I'm gonna delete and destroy what you've made". This attack was actually started a long time ago, even Sony doesn't know this attack is happening while it is going on. This attack collected a lot of information from Sony regarding their corporate files and secure data.
So we can learn that we never know what is going on in the back.

\begin{figure}[htb]
\centering
\includegraphics[width=9cm, height=6cm]{sonyimage.jpg}
\caption{Sony Attack Image}
\label{fig:sonyimage}
\end{figure}

\section{Forensic Science}
Doctor takes a lot of care while performing a surgery in order to prevent the baby from getting infections. So in that way when ever we log in to the websites which we think are really important for us, we should not save our passwords which will automatically fill next time when some one opens it. This reduces 80 percent of attacks.

\section{Subject Learned}
All the attacks which are going on now a days are mainly due to the lack of development perspective.So as a beginner in developing algorithms, what i think is the encryption standards must be high. By seeing the Sony attack, we never know what is going on around. What ever the things we do regarding the security, we must be very careful. one more thing i have learnt is think like an attacker. “Guess the next move of yours in an attacker perspective.”\\ I have also learned how to present and how to make the presentation interesting in the given time.












\end{document}
































Some intelligence techniques one must follow to avoid these type of Cyber threats are \\1. Information Security\\2. Intelligence Analysis\\3. Forensic Science